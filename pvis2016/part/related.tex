Our approach touches several research areas, including computation and visualization of protein cavities and their surfaces. 
Further we focus on the real-time visualization of surfaces and an ambient occlusion technique enhancing the depth perception of the molecular surface. 
Thus in this section we describe the existing techniques related to these topics.

\subsection{Extraction and Visualization of Cavities and Their Surfaces}
\cite{totrov1996contour}
\cite{krone2009interactive}
\cite{lindow2010accelerated}
\cite{parulek2012implicit}
\cite{connolly1983analytical}


\subsection{Order-Independent Transparency}
Current hardware enables implementation of well known approaches to order-independent transparency (OIT) such as Virtual Pixel Maps \cite{mammen1989transparency} or A-buffer \cite{carpenter1984abuffer}.
While rendering computer graphics with correct transparency has become possible in real-time already in the previous decade \cite{everitt2001interactive}, the topic is still hot nowadays.
There are appearing new techniques in the literature that are focusing on pushing the limits of previous methods, e.g., time complexity \cite{bavoil2008order}, memory requirements \cite{maule2012memory} \cite{vasilakis2015k+buffer} or precision \cite{enderton2011stochastic}.

\textcolor{red}{TODO OIT in molecules.}

%\cite{carpenter1984abuffer}
%\cite{everitt2001interactive}
%\cite{bavoil2008order}
\cite{yang2010real}
%\cite{enderton2011stochastic}
\cite{salvi2011adaptive}
%\cite{maule2012memory}%star
PUXELS \cite{kauker2013rendering}


\subsection{Ambient Occlusion}
\cite{tarini2006ambient}
\cite{grottel2012object}
\cite{staib2015ambient} 
\cite{borland2011ambient}




\begin{itemize}
  \item Performance drop of SES rendering on newer hardware (GF 680 GTX) --- we can perform better. We contacted authors, they do not know exactly why, they think it is change in the internal architecture between Fermi and Kepler (AJ)
  \item Slow rendering of SAS. Too many layers in pixels --- we can do better by surface layers detection (AJ)
\end{itemize}

There are more solutions taken from computer graphics; e.g., OIT (JP)

AOOM
\begin{itemize}
  \item We employ transparency modulation techniques presented in their paper \cite{borland2011ambient}
\end{itemize}

\subsection{Molecular Surface Representation}
\textcolor{red}{Shouldn't be this section part of motivation of our algorithm for SES? Also, it seems that it mixes two types of voids - cavities and pockets. It needs to be discussed.}

There are several types of molecular surface representation proposed in the literature~\cite{STAR2015}. 
However, for cavity analysis,  the solvent-excluded surface (SES) is the most commonly used representation used by the domain experts~\cite{todo}. 
This representation allows us to directly asses whether a solvent, approximated by a sphere of a given radius, is able to reach a binding site of interest on the molecular surface. 
Such binding site can be located inside a cavity or a tunnel, while the molecular surface might contain tens of cavities per single simulation time step. 
Additionally, computation of SES is not a trivial task, which also requires a substantial computation and algorithmic capabilities. 
Therefore, it would be essential to posses a technique that could provide us with instant computation and interactive and meaningful visualization of cavities in the context of molecular surface.


