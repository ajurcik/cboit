In this paper we presented a new approach for the real-time computation and visualization of the transparent molecular surface and exact representation of inner cavities. 
This was reached by introducing an improved representation of individual surface primitives, proposing a novel real-time algorithm for cavity extraction based on the solvent-excluded surface, and improving the performance of one of the existing approaches for visualization of transparent surfaces.
The usability of our solution was demonstrated on a case study which addressed the problem of interactive visual exploration of molecular dynamics.
The evaluation done by the domain experts revealed a possible extension of our solution which would remove the problem with perception of the ligand position with respect to highly transparent molecular surfaces.
Another possible extensions are related to the performance. 
For instance, using more tight bounds for ray-casting might further improve the performance. 
This holds especially for tori since each torus is {\textcolor{red}{ray-cast xxx (about 2) times in average. (AJ)
Moreover, the ray-casting might be done in OpenCL which would lower the bandwidth since the data that are shared amongst fragments could be fetched only once and shared using local memory.}
