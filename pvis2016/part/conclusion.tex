In this paper we presented our approach to the real-time computation and visualization of the transparent molecular surface and exact representation of inner cavities. 
This was reached by introducing the exact representation of individual surface primitives, proposing a novel real-time algorithm for cavity extraction based on the solvent-excluded surface, and improving the performance of one of the existing approaches to visualization of transparent surfaces.
The usability of our solution was demonstrated on a case study which addressed the problem of interactive visual exploration of molecular dynamics.
The evaluation done by the domain experts revealed a possible extension of our solution which would remove the problem with perception of the ligand position wrt. highly transparent molecular surface.
Another possible extensions are related to the performance. 
Using more tight bounds for ray-casting may further improve the performance. 
This holds especially for tori because each torus is {\textcolor{red}{ray-casted xxx (about 2) times in average. (AJ)
The ray-casting could be done using OpenCL which could lower bandwidth because the data common to many fragments could be fetched only once and shared using local (shared) memory. (AJ)
Transfer function.}
